%
% File poltext2016.tex
%
% Contact: jan.snajder@fer.hr or dsirinic@gmail.com
%%
%% Based on the style files for ACL-2015, which were, in turn,
%% Based on the style files for ACL-2014, which were, in turn,
%% Based on the style files for ACL-2013, which were, in turn,
%% Based on the style files for ACL-2012, which were, in turn,
%% based on the style files for ACL-2011, which were, in turn, 
%% based on the style files for ACL-2010, which were, in turn, 
%% based on the style files for ACL-IJCNLP-2009, which were, in turn,
%% based on the style files for EACL-2009 and IJCNLP-2008...

%% Based on the style files for EACL 2006 by 
%%e.agirre@ehu.es or Sergi.Balari@uab.es
%% and that of ACL 08 by Joakim Nivre and Noah Smith

\documentclass[11pt]{article}
\usepackage{poltext2016}
\usepackage{times}
\usepackage{url}
\usepackage{latexsym}
\usepackage{booktabs}
\usepackage{graphicx} % more modern
%\usepackage{epsfig} % less modern
\usepackage{subfigure}

\usepackage{natbib}
\usepackage{amsmath,amsfonts,amscd,amssymb}
\usepackage{dsfont}
\renewcommand{\vec}[1]{\mathbf{#1}}
\usepackage{hyperref}
\DeclareMathOperator*{\argmax}{argmax}
\DeclareMathOperator*{\argmin}{argmin}
\DeclareMathOperator*{\Corr}{Corr}
\newcommand{\R}{\mathds{R}}
\usepackage{multicol}
\usepackage{multirow}
\usepackage{pbox}
\usepackage{color}

% As of 2011, we use the hyperref package to produce hyperlinks in the
% resulting PDF.  If this breaks your system, please commend out the
% following usepackage line and replace \usepackage{icml2015} with
% \usepackage[nohyperref]{icml2015} above.
\usepackage{hyperref}

% Packages hyperref and algorithmic misbehave sometimes.  We can fix
% this with the following command.
\newcommand{\theHalgorithm}{\arabic{algorithm}}


\newcommand{\pola}[1]{\textcolor{blue}{#1}}
\newcommand{\felix}[1]{\textcolor{green}{#1}}

%\setlength\titlebox{5cm}

% You can expand the titlebox if you need extra space
% to show all the authors. Please do not make the titlebox
% smaller than 5cm (the original size); we will check this
% in the camera-ready version and ask you to change it back.

\title{Predicting political party affiliation from text}

\author{Felix Biessmann\thanks{~\tt felix.biessmann@gmail.com}\\
    \And
 Pola Lehmann\thanks{ ~{\tt pola.lehmann@wzb.eu} }\\
%  WZB f\"ur Sozialforschung\\
\And 
Daniel Kirsch
 \And
  Sebastian Schelter \thanks{~\tt sebastian.schelter@tu-berlin.de}\\ 
%  TU Berlin
}
% Sieht irgendwie komisch aus, mit nur zwei angegebenen affiliations, und meine ist ja auch so lang. Sollen wir die nicht in die Fussnote zur email Adresse packen?

\date{}

\begin{document}
\maketitle

\begin{abstract}
Every day large amounts of text are produced in public discourse. Some of this text is produced by actors whose political colour is very obvious. But many actors cannot be clearly attributed to a political party, yet their statements might be biased towards a specific party. Identifying such biases is crucial for political research as well as media consumers, especially when analysing the influence of the media on political discourse and vice versa. In this study we investigate to what extent political party affiliation can be predicted from textual content. Results indicate that automatic classification of political affiliation is possible with above chance accuracy also across text domains. We propose methods to better interpret these results and find that features not related to political policies, such as speech sentiment, can be discriminative and thus exploited by text analysis models. 
\end{abstract}

\section{Introduction}
\label{sec:intro}
%
Analysis and classifications of political text is and has been a very important tool to generate political science data \cite{Benoit.Forthcoming}. Traditionally such classifications are done by experts, who read and label the text of interest.\footnote{See for example the Manifesto Project, the Comparative Agendas Project or Poltext.} This is, however, a very time consuming task and thus sets various limits to the possible amount of data that a few experts can analyze. The growing field of automated text analysis, that allows the analysis of much more text in less time, is therefore of great interest to political scientists. Additionally automated text analyses allow for a more objective and replicable analysis of political text then human coders could achieve \cite{Benoit.2}.

A major problem with automated text analyses is generalisation to text domains other than that on which the system was trained \cite{Slapin.2014}. 
%The political experts who judge and label political texts have read enough content of different domains to detect political bias in a variety of contexts and styles. 
% Den vorangegangenen Satz würde ich streichen, weil ich mit der Aussage nicht übereinstimme. Das ist wahrscheinlich wieder der Unterschied der Disziplinen, ihr müsst rechtfertigen, warum Computer es können, wenn Menschen es können, wir müssen in der Regel rechtgertigen warum Menschen es können, wenn Computer es besser machen. 
% Ich glaube, dass Computer es besser machen, wuerde niemand behaupten. Computer machen das genau so gut oder schlecht, wie die qualitaet der daten (die ja von menschen gelabeled oder selektiert werden) ihnen erlaubt. Der Algorithmus spiel auch ne rolle, aber das hauptproblem ist Daten und die koennen nur von Menschen kommen. Computer sind hilfreich als assistive technology, weil sie das ganze skalierbar machen. 
Machine learning algorithms are naturally prone to poor generalisation performance if the training data is biased towards one text domain. Unfortunately good unbiased training data is difficult to obtain. One of the best sources for automated political text analysis systems are plenary debates of the parliament: Many studies are based on this type of data as it is a large source of text that can be clearly associated with a party. Here we examine to what extent models trained on this data can be generalised to other text domains, such as party manifestos and texts from social media. 
We discuss the effects of text length and domain shift of text data and investigate some potential reasons for the differences in classification performance. 

We investigate predictions of the models with three strategies. First the model misclassifications are related to changes in party policies. Second sentiment analysis is used to investigate whether this aspect of language has discriminatory power. Third univariate measures of correlation between text features and party affiliation allow to relate the predictions to the kind of information that political experts use for interpreting texts. 

In the following, \autoref{sec:data} gives an overview of the data acquisition and preprocessing methods, \autoref{sec:model} presents the model, training and evaluation procedures; in \autoref{sec:results} the results are discussed and \autoref{sec:conclusion} concludes with some interpretations of the results.

\section{Data Sets and Feature Extraction}\label{sec:data}
%
All experiments were run on publicly available data sets of German political texts and standard libraries for processing the text. The following sections describe the details of data acquisition and feature extraction.

\subsection{Data}
Annotated political text data was obtained from three sources: a) the plenary debates held in the German parliament ({\em Bundestag}) b) all manifesto texts of parties winning seats in the election to the German parliament and c) facebook page posts of all parties. The texts from plenary debates were used to train a classifier and evaluate it on this in-domain data. The latter two data sources were used to test the generalization performance of the classifier on out-of-domain data. 

\paragraph{Parliament discussion data} Parliament texts are annotated with the respective party label. The protocols of plenary debates are available through the website of the German Bundestag \cite{bundestag}; an open source API was used to query the data in a cleaned and structured format \cite{bundestag-github}. Each uninterrupted part was treated as a separate speech. 

\paragraph{Party manifesto data}
The party manifesto text was taken from the Manifesto Corpus  \cite{manifesto}. The data released in this project mainly comprises the complete manifestos of all parties that have won seats at a national election. Each statement or {\em quasi-sentence}\footnote{A quasi-sentence has the length of an argument. It is never longer than one sentence.} is annotated with one of 56 policy issue categories. Examples for the policy categories are {\em welfare state expansion, welfare state limitation, democracy, equality}; for a complete list and detailed explanations on how the annotators were instructed see \cite{leftright}. Each quasi-sentence has  two types of labels: the party affiliation and the manually assigned policy issue aimed at in each quasi-sentence. The length of each annotated statement in the party manifestos is rather short. The median length is 95 characters or 12 words.\footnote{The longest statement is 522 characters (65 words) long, the 25\%/50\%/75\% percentiles are 63/95/135 characters or 8/12/17 words, respectively.} 
In order to increase the length of the texts used for classification the policy labels were used to aggregate the data into the following topics: {\em External Relations, Freedom and Democracy, Political System, Economy, Welfare and Quality of Life, Fabric of Society, Social Groups}. So in this setting each party had just one data point for each of the topics. 

\paragraph{Facebook post data}
For each party we crawled their facebook pages \cite{gruene-fb, spd-fb, cducsu-fb, linke-fb} and extracted the post text, excluding all comments and other information. Like the manifesto data, these texts are very short. As aggregation per topic was not possible for this data, we aggregated the texts by splitting all texts into parts of 1000 words. 

\subsection{Bag-of-Words Vectorization}\label{sec:bow-vectorization}
All text data was tokenised and transformed into bag-of-word (BOW) vectors as implemented in scikit-learn \cite{scikit-learn}. Several options for BOW vectorizations were tried, including term-frequency-inverse-document-frequency normalisation, n-gram patterns up to size $n=3$ and different cutoffs for discarding too frequent and too infrequent words.

\section{Classification Model and Training}\label{sec:model}
Bag-of-words feature vectors were used to train a multinomial logistic regression model. Let $y\in\{1,2,\dots,K\}$ be the true party affiliation and $\vec{w}_1,\dots,\vec{w}_K\in\R^{d}$ the weight vectors associated with the $k$th party then the party affiliation estimate is modelled as
\begin{eqnarray}\label{eq:logreg_multiclass}
p(y=k|\vec{x}) = \frac{e^{z_k}}{\sum_{j=1}^K e^{z_j}}  \textrm{ with }  z_k=\vec{w}_k^{\top}\vec{x}.
\end{eqnarray}

\subsection{Optimisation of Model Parameters}\label{sec:crossvalidation}
The model pipeline contained a number of  hyperparameters that were optimised using gridsearch cross-validation. To this end the parliament speech data was split into training and validation set in a 90\%/10\% ratio; the pipeline was trained with each parameter setting on the training set and its performance validated on the validation set. The parameters of the best performing model were then used to train a model on the training and validation set data. None of the data in the separately held back in-domain test data nor the out-of-domain test data sets was used for this hyperparameter optimization. 
%
\subsection{Sentiment analysis}\label{sec:sentiment_analysis_methods}
A publicly available key word list was used to extract sentiments \cite{remquahey2010}. A sentiment vector $\vec{s}\in\R^d$ was constructed from the sentiment polarity values in the sentiment dictionary. The sentiment index used for attributing positive or negative sentiment to a text was computed as the cosine similarity between BOW vectors and sentiment vector.

\subsection{Interpreting bag-of-words models}\label{sec:correlations_methods}
Interpreting coefficients of linear models (independent of the regularizer used) implicitly assumes uncorrelated features; this assumption is violated by the text data used in this study. Thus direct interpretation of the model coefficients $\vec{w}_k$ is problematic, see also \cite{Zien2009, Haufe2013}. In order to allow for better interpretation of the predictions and to assess which features are discriminative, correlation coefficients between each word and the party affiliation label were computed. 

\section{Results}\label{sec:results}

The following section gives an overview of the results for all political bias prediction tasks. 
Predictions compared with the manifesto data were done using models trained on texts from the 17th Bundestag, predictions obtained for facebook post texts were computed with models trained on the 18th Bundestag.\footnote{We used the speeches from the 17th legislative period for the first task as this legislature is already completed and offers more data. Results for the 18th Bundestag are similar but omitted for brevity. We used the speeches of the 18th legislative period for the facebook posts as the posts were more recent.}
% Frage: Wollen wir nicht schreiben, dass wir das erste auf 17 und 18 gerechnet haben, aber nur 17. reporten? Stimmt ja und macht die Ergebnisse vielleicht vertrauensvoller. Einen entsprechenden Satz hatte ich ja schonmal formuliert.  

\subsection{In-domain predictions}

When predicting party affiliation on text data from the same domain that was used for training the model average precision and recall values of above 0.6 are obtained. The evaluation results for the political party affiliation prediction on in-domain data (held-out parliamentary speech text) for the 17th Bundestag are listed in \autoref{tab:results_in-domain}.
These results are comparable to those of \cite{Hirst2014} who report a classification accuracy of 0.61 on a five class problem predicting party affiliation in the European parliament.


\begin{table}[t]
\caption{
\label{tab:results_in-domain}
{\bf In-domain classification performance} for data from the 17th legislative period on in-domain data. $N$ denotes number of data points in the evaluation set.
}
\begin{center}
\begin{tabular}{lcccc}
    &         precision    &recall &  f1-score  & N  \\
\hline \hline
       cducsu   &    0.62  &    0.81  &    0.70  &     706\\
        fdp    &   0.70   &   0.37  &    0.49    &   331\\
     gruene &      0.59  &    0.40   &   0.48   &    298\\
      linke    &   0.71   &   0.61  &    0.65    &   338\\
        spd   &    0.60   &   0.69  &    0.65   &    606\\
\hline
 total &      0.64   &   0.63   &   0.62    &  2279 
%
\end{tabular}
\end{center}
\end{table}


\subsection{Out-of-domain predictions}
% Validation on manifesto data hat mich hier irritiert, weil es in dem Absatz ja auch um die facebook Daten geht. 
% Sorry, das war schlecht strukturiert - wollte validation als ueberschrift drinlassen, weil das von Dir war, Pola.
For out-of domain data obtained from manifesto data the models yield significantly lower precision and recall values between 0.3 and 0.4, see \autoref{tab:results_out-of-domain}. For the facebook post data a similar effect was observed. The short texts resulted in poor prediction accuracies of 0.51 on average. In addition in this setting classes were highly unbalanced since some parties have an order of magnitude more posts than others. 


\begin{table}[t]
\caption{
\label{tab:results_out-of-domain}
{\bf Out-of-domain} classification performance (quasi-sentence level) on {\bf manifesto data} of a classifier trained on speeches of the 17th legislative period of the Bundestag.
}

\begin{center}
\begin{tabular}{lcccc}
    &         prec.    &recall &  f1-score  & N  \\
\hline \hline
    cducsu    &   0.26   &   0.58   &   0.36    &   2030 \\
    fdp    &   0.38   &   0.28   &   0.33    &   2319 \\
     gruene   &    0.47    &  0.20   &   0.28    &  3747\\
      linke     &  0.30  &    0.47    &  0.37    &   1701\\
        spd     &  0.26  &    0.16   &   0.20    &   2278\\
\hline
total    &   0.35  &    0.31  &    0.30   &   12075\\
%
\end{tabular}
\end{center}

\end{table}


\begin{table}[t]
\caption{
\label{tab:results_topic}
{\bf Out-of-domain} classification performance (topic level) on {\bf manifesto data}. Compared to quasi-sentence level predictions (\autoref{tab:results_out-of-domain}) the predictions made on topic level are more reliable.}
\begin{center}
\begin{tabular}{lcccc}
    &         precision    &recall &  f1-score  & N  \\
    \hline
        \hline
cducsu     &  0.64  &    1.00  &    0.78    &     7\\
       fdp    &   1.00    &  1.00    &  1.00    &     7\\
    gruene  &     1.00  &    0.86  &    0.92    &     7\\
     linke    &   1.00   &   1.00     & 1.00    &     7\\
       spd   &    0.80   &   0.50    &  0.62     &    8\\
    \hline
total  &     0.88   &   0.86   &   0.86  &      36\\
\end{tabular}
\end{center}
\end{table}


\subsection{Influence of text length on accuracy}
A main factor that made the prediction on the out-of-domain prediction task particularly difficult is the short length of the texts to be classified, see also \autoref{sec:data}. In order to investigate the effect of text length the data was aggregated into longer texts. Manifesto data was aggregated into political topics. The topic level prediction results are shown in \autoref{tab:results_topic}. For all parties except for the SPD F1 scores of above 0.8 are obtained. As there were no topic labels for the facebook posts, those texts were aggregated by sampling 50 texts of 1000 words each of the concatenated facebook posts of a party. The results are shown in \autoref{tab:results_fb}. Also for these texts prediction accuracies comparable to the in-domain case can be achieved. This increase is in line with previous findings on the influence of text length on political bias prediction accuracy \cite{Hirst2014}. 

\begin{table}[t]
\caption{
\label{tab:results_fb}
{\bf  Out-of-domain} classification performance on 50 randomly selected {\bf facebook posts} of respective party (text length: 1000 words). The average prediction performance is comparable to that on in-domain test data.}
\begin{center}
\begin{tabular}{lcccc}
    &         precision    &recall &  f1-score  & N  \\
    \hline
        \hline
 cducsu     &  0.65     & 1.00  &    0.79     &   50\\
     gruene   &    0.67   &   0.12  &    0.20   &     50\\
      linke       &0.60    &  0.82    &  0.69    &    50\\
        spd       &1.00 &     0.92   &   0.96    &    50\\
\hline
avg / total    &   0.73   &   0.71  &    0.66   &    200\\
\end{tabular}
\end{center}

\end{table}


\paragraph{Misclassifications and policy change}
An explanation for misclassifications could be that parties change their policy positions. This requires inspecting the specific data points that were misclassified. The confusion matrix for the 17th Bundestag in \autoref{tab:confusion_topic} shows that on the topic level the SPD manifesto texts are often predicted as belonging to the CDU/CSU. A similar effect can be observed in the 18th Bundestag when predicting both manifesto texts as well as facebook posts (results not shown): texts of the Green party are often predicted to be from the CDU/CSU. One explanation for this misclassification could be policy changes. \\

% Confusion matrix topic level
\begin{table}[t]\label{tab:conf_mat_four_class}
\caption{\label{tab:confusion_topic} {\bf Topic level confusion matrices} of manifesto texts.}
\vspace{0.5em}
\begin{tabular}{lc|ccccc}
&& \multicolumn{5}{c}{\bf Predicted}\\
&& cducsu & fdp& gruene& linke& spd\\
\hline
\multirow{5}{*}{\rotatebox{90}{\pbox{2cm}{\centering {\bf True}}}} &cducsu &7& 0& 0& 0& 0\\
&fdp&0& 7& 0& 0& 0\\
&gruene&0& 0& 6& 0& 1\\
&linke&0& 0& 0& 7& 0\\
&spd&4& 0& 0& 0& 4\\
\end{tabular}
\end{table}


\begin{table*}[t]
\caption{
\label{tab:results_binary_17}
Classification accuracy on the binary prediction problem, categorizing texts into government and opposition. Out-of-domain accuracy again drops close to chance performance for the manifesto data but remains higher for the facebook post texts. 
}
\begin{center}
\begin{tabular}{lccc}
& {\bf In-Domain} & \multicolumn{2}{c}{{\bf Out-of-Domain}}\\
& Parliament & Manifestos & Facebook Posts\\
\hline
Accuracy    &   0.88   &   0.60&      0.76\\
%
\end{tabular}
\end{center}
\end{table*}

\subsection{Predicting government status}\label{sec:sentiment_result}
For a better comparison with other studies that predict party affiliation in a two party system we also trained a model on government membership labels. In \autoref{tab:results_binary_17} the results are shown for the 17th legislative period. While the in-domain prediction accuracy is close to 0.9, the out-of-domain evaluation on manifesto data drops again close to chance performance. This is in line with the results on binary classification of political bias in the Canadian parliament \cite{Yu2008}. The authors report classification accuracies between 0.80 and 0.87 and find a pronounced drop in performance on texts from a different domain (e.g. older texts or texts from another chamber). In our results aggregation into topics did not increase the accuracy in this binary setting when classifying manifesto texts. The drop in accuracy when applying the binary classifier on facebook data (aggregated as in the party affiliation case) was less pronounced, accuracies were above 0.70. 

\subsection{Discriminative features}
\label{sec:discrim_effect}
An important question is whether the difference between the features of each party relies on different policies or on on other aspects of the text. To get more insight into what aspects are discriminative for government membership and for parties we analysed, first, the effect of text sentiment and the text features most correlated with each party.

% ich glaub der uebergang von dem sentiment in der letzten subsection passt ganz gut zu den sentiment analysen im naechsten
% auch, weils beim sentiment noch um die binary classification geht, die gerade kam
\paragraph{Sentiment correlates with political power}
The drop in prediction accuracy in the government prediction task was more pronounced for manifesto texts than for facebook posts. What do facebook posts and plenary debates have in common? Both the speakers in the parliament as well as the authors of the facebook posts know which party is in the government -- while the authors of the manifesto texts do not know that as these texts were written before the elections. A language feature that could capture this is sentiment. 
In order to assess this effect we quantified the correlation between government membership as well as number of seats in the parliament with text sentiment. Our results in  \autoref{tab:sentiments} show that government membership correlates with positive sentiment with a correlation coefficient of 0.98 and the number of seats correlates with 0.89. Also previous studies find that text features discriminative in a two party system are not necessarily related to policies but more to language of defence and attack \cite{Hirst2014}. 

\begin{table}[t]
\caption{
\label{tab:sentiments}
Correlation coefficient between average sentiment of political speeches of a party in the german Bundestag with two indicators of political power, a) membership in the government and b) the number of seats a party occupies in the parliament.
}
\begin{center}
\begin{tabular}{lcc}
   Sentiment vs. &          Gov. Member    &  Seats\\
\hline\hline
17th Bundestag    &  0.84 & 0.70\\
18th Bundestag   &  0.98 & 0.89\\
%
\end{tabular}
\end{center}
\end{table}


%
\paragraph{Correlations between words and parties}
To determine other discriminative features we quantified which words were preferentially used by each party by measuring the correlation of single words with the party label. Some unspecific stopwords are excluded. We find clear differences between the parties and these difference are in line with the parties ideologies. 
\paragraph{\bf Left party (linke)} Often used words include referrals to big companies ({\em konzerne}) and their profits ({\em profite}), the working class {\em beschaeftigte}, the social welfare program {\em hartz iv} as well as war ({\em krieg}).
\paragraph{\bf Green party (gruene)} Use words related to environmental damage ({\em klimaschaedlichen}), exploited low wage employees ({\em leiharbeitskraefte}) and pensions ({\em garantierente}).
\paragraph{\bf Social Democratic Party (SPD)} Uses mostly unspecific words related to the parliament and governmental processes ({\em staatssekretaerin, kanzlerin, bundestagsfraktion}) and some words related to cutting of expenses ({\em kuerzungen}).
\paragraph{\bf Christian Democratic Union/Christian Social Union (CDU/CSU)}
Often used words relate to a pro-economy attitude, such as competitiveness or (economic) development ({\em wettbewerbsfaehigkeit, entwicklung}) and words related to security ({\em sicherheit, stabilitaet}). 

\section{Conclusions and Limitations}\label{sec:conclusion}
Evaluating classifiers trained on parliament speeches we find that automated political bias prediction is possible with above chance accuracy even beyond the training text domain. These results suggest that such systems could be helpful as assistive technology for instance for human annotators in an active learning setting. 

In line with other findings \cite{Yu2008, Hirst2014}, we find a strong effect of text length and text domain on the generalization performance of the classifier. The first effect, longer texts are easier to classify, makes intuitive sense. Also humans are worse at judging the political bias of shorter texts out of context: \cite{Benoit.Forthcoming}, for example, found in an experiment where experts where differentiating whether specific sentences were dealing with economic or social policy only about 35\% agreement between all expert coders. However short texts are a realistic challenge for automated political bias prediction systems: Often political texts from social media data and other web media are indeed very short. But political education can benefit from automatic analyses of these data streams, as these media have strong influence on public opinions yet are difficult to analyse with human annotators due to the large volume of data. \\
The second effect, i.e. the drop in generalization performance on out-of-domain data, can be alleviated by aggregating texts into longer segments. In the case of party affiliation prediction, the out-of-domain classification is on a par or even better than the prediction accuracy on in-domain data. However in the binary classification setting (government membership), aggregating manifesto data, which was written without the knowledge of which party would be member of the government, into longer texts does not counteract the effect of out-of-domain classification. We attribute this effect in part to the fact that sentiment appears to be a discriminative feature for government membership. 

\subsection*{Acknowledgements}
We would like to thank Friedrich Lindenberg for factoring out the \url{https://github.com/bundestag/plpr-scraper} from his Bundestag project. Michael Gaebler provided helpful feedback on an earlier version of the manuscript. \\
%
\bibliographystyle{plain}
\bibliography{political_bias_prediction}


\end{document}
